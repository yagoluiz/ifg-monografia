\chapter{Conclusão}
\label{cap:conclusao}

A cidade de Luziânia possui atualmente um grande volume de informações relacionadas a serviços básicos a população, compostos pelos serviços de: educação, lazer, saúde e segurança. Todas as informações são organizadas em documentos físicos e planilhas eletrônicas, dificultando o acesso posterior e ocasionando possíveis inconsistências no armazenamento e na recuperação dessas informações. Para resolver esse problema, foram propostos as seguintes soluções: a criação de um banco de dados geográfico para armazenamento dos dados que possuam localização geográfica e a criação de um sistema de informação geográfico para organização e apresentação dos dados de forma georreferenciada.

Primeiramente, foi realizado o processo de coleta e organização das informações, sendo posteriomente adicionadas ao banco de dados. Em seguida, começou a implementação de um SIG Web. Todo o processo de implementação do sistema fez uso de ferramentas livres para criação.

Em cumprimento dos objetivos deste trabalho, foi desenvolvido o SIG Web Luziânia com o objetivo de organizar, gerenciar e visualizar dados relacionados aos serviços de educação, lazer, saúde e segurança. O padrão OMT-G foi utilizado na criação do modelo de dados, a extensão espacial utilizada no banco de dados geográfico foi o PostGIS, e por fim, as principais tecnologias utilizadas no desenvolvimento do SIG Web Luziânia foram: C\#, \textit{ASP.NET MVC}, HTML5, Javascript e o \textit{Leaflet}.

Como vantagens e benefícios para a cidade de Luziânia, o SIG Web Luziânia apresenta as seguintes perspectivas:

\begin{itemize}
\item Organiza as informações evitando inconsistência;
\item Facilita o gerenciamento das informações pelos órgãos competentes;
\item Transparência das informações para a população de Luziânia; 
\item Apresenta os dados de forma georreferenciada;
\item Possibilita a criação de novas funcionalidades e serviços;
\item Possibilita o acesso as informações em dispositivos móveis e \textit{desktops} a partir do \textit{browser}.
\end{itemize}

Como trabalhos futuros, pretende-se ter o cadastro de mais informações, a criação de novos serviços, disponibilização dos dados do sistema para \textit{download} e publicação do sistema na Internet.